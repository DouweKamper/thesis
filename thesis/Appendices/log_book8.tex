\subsection{Change number of observations}\label{nr obs}
In previous sections different problems were resolved. In this section these changes are combined. This includes correcting the added noise to observations (\hyperref[check noise]{\textcolor{blue}{Section }\ref{check noise}}), using a virtual environment (\hyperref[virtual environment]{\textcolor{blue}{Section }\ref{virtual environment}}), changing settings for MODFLOW to prevent simulation errors (\hyperref[MODFLOW_ERRORS]{\textcolor{blue}{Section }\ref{MODFLOW_ERRORS}}) and parallelising emcee (\hyperref[par emcee]{\textcolor{blue}{Section }\ref{par emcee}}). Additionally, DEsnooker now uses 90\% DE moves and 10\% snooker update moves, similar to \cite{terbraak2008differential}. With these changes in mind, the number of measurement locations (observations) is varied, investigating the effect on the posterior.  

In \hyperref[emcee first results]{\textcolor{blue}{Section }\ref{emcee first results}}, a total of 3 observations were used in every MODFLOW simulation. These 3 observations were selected by the author in such a way that they are spread out throughout the xy grid of each model. The main drawback of this approach is that only 3 observations were selected, which means more parameters exist than measurements for Model 5, possibly leading to equifinality. Additionally, all models but Model 1 lack observations in some layers because the observations are limited to the upper most layer only.

The new number of measurement locations are 1, 3 and 5 locations per layer, with the (x,y) coordinates being consistent across layers (measurement well style). This means that with only 1 measurement location per layer, Model 4 already increases from 3 to 5 total observations. These coordinates were selected randomly with the function \textit{randint} from the built-in \textit{random} module. To compare the effect of different number of sampling locations on the posterior, the posterior mean and standard deviation are presented in \hyperref[log8 mean]{\textcolor{blue}{Section }\ref{log8 mean}}.

\subsubsection{short runs}\label{log8 mean}
For Model 1, the ensemble means shifts towards the true optimum when the number of observations increases to 5, consistent for all algorithms (Table \ref{tab_logbook8_obs1_model1}, \ref{tab_logbook8_obs3_model1} \& \ref{tab_logbook8_obs5_model1}). With all chains having 1500 steps out of which 500 from burn-in are subtracted, leaving 1000 steps contributing to the main sampling presented in the tables. 

For Model 2, the ensemble means also shift towards the true optimum with increasing number of observations (Table \ref{tab_logbook8_obs1_model2}, \ref{tab_logbook8_obs3_model2} \& \ref{tab_logbook8_obs5_model2}). However, when using 5 observation locations, DE does not (yet) converge to the true optimum for $\theta_2$ (Table \ref{tab_logbook8_obs5_model2}). 

For Model 3, the ensemble means also shift towards the true optimum with increasing number of observations (Table \ref{tab_logbook8_obs1_model3}, \ref{tab_logbook8_obs3_model3} \& \ref{tab_logbook8_obs5_model3}). For Model 3, using 5 observation locations, DE does not (yet) converge to the true optimum for $\theta_2$ (Table \ref{tab_logbook8_obs5_model3}), which is similar to Model 2 (Table \ref{tab_logbook8_obs5_model2}).

For Model 4, increasing the number of observations from 1 to 5 appears to not improve parameter calibration (Table \ref{tab_logbook8_obs1_model4}, \ref{tab_logbook8_obs3_model4} \& \ref{tab_logbook8_obs5_model4}), contrary to the other models.

\subsubsection{long runs}
No method converged to the true optima for Model 4 with a chain length of 1500 steps and increasing the number of observations per layer from 1 to 5 also showed no improvement. An ensemble with longer chains of 10000 steps and 50 observations per layer was run in an attempt to converge to the true optima. 

%flaw is that multiple instances of the same location are possible. -> should save file with locations

% read how float environments work in latex
% also read if changes have been made since publication 2014
\clearpage

% Table 1
\begin{table}[ht]
\caption{Ensemble mean ($\mu$) and the standard deviation of the ensemble mean ($\sigma$), presented as mean (std). Results are from calibrating the parameter ($\theta$) of Model 1 with different samplers (Stretch, DE, DEsnooker), with each three ensembles (e). One observation was used.}
\label{tab_logbook8_obs1_model1}
\begin{tabularx}{\textwidth}{XX}
\toprule
 & $\theta_1$ \\
\midrule
Stretch (e1) & 10.3 (1.1) \\
Stretch (e2) & 10.8 (1.5) \\
Stretch (e3) & 10.3 (2.0) \\
\midrule
DE (e1) & 10.6 (0.8) \\
DE (e2) & 10.7 (0.5) \\
DE (e3) & 10.7 (0.5) \\
\midrule
DEsnooker (e1) & 11.0 (0.6) \\
DEsnooker (e2) & 11.0 (0.5) \\
DEsnooker (e3) & 10.7 (0.4) \\
\midrule
\textbf{True value} & \textbf{5.0} \\
\bottomrule
\end{tabularx}
\end{table}


% Table 2
\begin{table}[ht]
\caption{Ensemble mean ($\mu$) and the standard deviation of the ensemble mean ($\sigma$), presented as mean (std). Results are from calibrating the parameter ($\theta$) of Model 1 with different samplers (Stretch, DE, DEsnooker), with each three ensembles (e). Three observations were used.}
\label{tab_logbook8_obs3_model1}
\begin{tabularx}{\textwidth}{XX}
\toprule
 & $\theta_1$ \\
\midrule
Stretch (e1) & 11.0 (0.7) \\
Stretch (e2) & 11.5 (1.6) \\
Stretch (e3) & 10.6 (0.9) \\
\midrule
DE (e1) & 11.2 (0.5) \\
DE (e2) & 10.8 (0.4) \\
DE (e3) & 10.9 (0.4) \\
\midrule
DEsnooker (e1) & 11.0 (0.6) \\
DEsnooker (e2) & 11.0 (0.6) \\
DEsnooker (e3) & 11.2 (0.9) \\
\midrule
\textbf{True value} & \textbf{5.0} \\
\bottomrule
\end{tabularx}
\end{table}


% Table 3
\begin{table}[ht]
\caption{Ensemble mean ($\mu$) and the standard deviation of the ensemble mean ($\sigma$), presented as mean (std). Results are from calibrating the parameter ($\theta$) of Model 1 with different samplers (Stretch, DE, DEsnooker), with each three ensembles (e). Five observations were used.}
\label{tab_logbook8_obs5_model1}
\begin{tabularx}{\textwidth}{XX}
\toprule
 & $\theta_1$ \\
\midrule
Stretch (e1) & 5.6 (0.1) \\
Stretch (e2) & 5.6 (0.1) \\
Stretch (e3) & 5.6 (0.1) \\
\midrule
DE (e1) & 5.6 (0.0) \\
DE (e2) & 5.6 (0.0) \\
DE (e3) & 5.6 (0.0) \\
\midrule
DEsnooker (e1) & 5.6 (0.0) \\
DEsnooker (e2) & 5.6 (0.0) \\
DEsnooker (e3) & 5.6 (0.0) \\
\midrule
\textbf{True value} & \textbf{5.0} \\
\bottomrule
\end{tabularx}
\end{table}



% Table 4
\begin{table}[ht]
\caption{Ensemble mean ($\mu$) and the standard deviation of the ensemble mean ($\sigma$), presented as mean (std). Results are from calibrating the parameters ($\theta$) of Model 2 with different samplers (Stretch, DE, DEsnooker), with each three ensembles (e). One observation was used}
\label{tab_logbook8_obs1_model2}
\begin{tabularx}{\textwidth}{XXX}
\toprule
 & $\theta_1$ & $\theta_2$ \\
\midrule
Stretch (e1) & 1.2 (0.3) & 5.4 (2.3) \\
Stretch (e2) & 1.5 (0.4) & 4.3 (1.5) \\
Stretch (e3) & 1.5 (0.4) & 4.3 (1.9) \\
\midrule
DE (e1) & 1.4 (0.3) & 4.4 (1.6) \\
DE (e2) & 1.3 (0.3) & 5.0 (1.7) \\
DE (e3) & 1.4 (0.3) & 4.9 (1.7) \\
\midrule
DEsnooker (e1) & 1.3 (0.2) & 4.5 (1.2) \\
DEsnooker (e2) & 1.6 (0.2) & 3.0 (0.6) \\
DEsnooker (e3) & 1.5 (0.4) & 4.2 (1.3) \\
\midrule
\textbf{True value} & \textbf{2.0} & \textbf{1.0} \\
\bottomrule
\end{tabularx}
\end{table}



% Table 5
\begin{table}[ht]
\caption{Ensemble mean ($\mu$) and the standard deviation of the ensemble mean ($\sigma$), presented as mean (std). Results are from calibrating the parameter ($\theta$) of Model 2 with different samplers (Stretch, DE, DEsnooker), with each three ensembles (e). Three observations were used.}
\label{tab_logbook8_obs3_model2}
\begin{tabularx}{\textwidth}{XXX}
\toprule
 & $\theta_1$  & $\theta_2$ \\
\midrule
Stretch (e1) & 1.3 (0.3) & 2.3 (2.0) \\
Stretch (e2) & 1.4 (0.2) & 1.7 (0.5) \\
Stretch (e3) & 1.4 (0.2) & 1.7 (0.4) \\
\midrule
DE (e1) & 1.4 (0.1) & 1.6 (0.2) \\
DE (e2) & 1.4 (0.1) & 1.7 (0.2) \\
DE (e3) & 1.3 (0.4) & 3.2 (4.9) \\
\midrule
DEsnooker (e1) & 1.5 (0.1) & 1.6 (0.2) \\
DEsnooker (e2) & 1.4 (0.1) & 1.6 (0.2) \\
DEsnooker (e3) & 1.4 (0.1) & 1.6 (0.1) \\
\midrule
\textbf{True value} & \textbf{2.0} & \textbf{1.0}\\
\bottomrule
\end{tabularx}
\end{table}


% Table 6
\begin{table}[ht]
\caption{Ensemble mean ($\mu$) and the standard deviation of the ensemble mean ($\sigma$), presented as mean (std). Results are from calibrating the parameter ($\theta$) of Model 2 with different samplers (Stretch, DE, DEsnooker), with each three ensembles (e). Five observations were used.}
\label{tab_logbook8_obs5_model2}
\begin{tabularx}{\textwidth}{XXX}
\toprule
 & $\theta_1$  & $\theta_2$ \\
\midrule
Stretch (e1) & 2.3 (0.0) & 0.6 (0.0) \\
Stretch (e2) & 2.4 (0.0) & 0.6 (0.1) \\
Stretch (e3) & 2.4 (0.0) & 0.6 (0.0) \\
\midrule
DE (e1) & 2.1 (0.5) & 3.0 (6.9) \\
DE (e2) & 2.2 (0.5) & 2.8 (6.5) \\
DE (e3) & 2.2 (0.5) & 2.4 (5.5) \\
\midrule
DEsnooker (e1) & 2.4 (0.0) & 0.6 (0.0) \\
DEsnooker (e2) & 2.4 (0.0) & 0.6 (0.0) \\
DEsnooker (e3) & 2.4 (0.0) & 0.6 (0.0) \\
\midrule
\textbf{True value} & \textbf{2.0} & \textbf{1.0}\\
\bottomrule
\end{tabularx}
\end{table}
\clearpage



% Table 7
\begin{table}[ht]
\caption{Ensemble mean ($\mu$) and the standard deviation of the ensemble mean ($\sigma$), presented as mean (std). Results are from calibrating the parameters ($\theta$) of Model 3 with different samplers (Stretch, DE, DEsnooker), with each three ensembles (e). One observation was used}
\label{tab_logbook8_obs1_model3}
\begin{tabularx}{\textwidth}{XXXX}
\toprule
 & $\theta_1$ & $\theta_2$ & $\theta_3$\\
\midrule
Stretch (e1) & 1.2 (0.3) & 11.4 (2.0) & 5.6 (2.1) \\
Stretch (e2) & 1.2 (0.3) & 12.0 (1.8) & 6.1 (2.5) \\
Stretch (e3) & 1.5 (0.7) & 10.9 (1.1) & 4.9 (2.6) \\
\midrule
DE (e1) & 1.2 (0.3) & 11.9 (1.5) & 6.8 (2.1) \\
DE (e2) & 1.2 (0.3) & 10.9 (1.0) & 6.1 (3.2) \\
DE (e3) & 1.3 (0.3) & 11.9 (0.9) & 5.8 (1.3) \\
\midrule
DEsnooker (e1) & 1.3 (0.3) & 10.9 (1.7) & 5.7 (1.4) \\
DEsnooker (e2) & 1.5 (0.5) & 11.6 (1.3) & 5.6 (2.2) \\
DEsnooker (e3) & 1.2 (0.3) & 11.9 (1.6) & 5.8 (1.7) \\
\midrule
\textbf{True value} & \textbf{1.0} & \textbf{0.01} &\textbf{10.0} \\
\bottomrule
\end{tabularx}
\end{table}

% Table 8
\begin{table}[ht]
\caption{Ensemble mean ($\mu$) and the standard deviation of the ensemble mean ($\sigma$), presented as mean (std). Results are from calibrating the parameters ($\theta$) of Model 3 with different samplers (Stretch, DE, DEsnooker), with each three ensembles (e). Three observations were used}
\label{tab_logbook8_obs3_model3}
\begin{tabularx}{\textwidth}{XXXX}
\toprule
 & $\theta_1$ & $\theta_2$ & $\theta_3$\\
\midrule
Stretch (e1) & 1.4 (0.3) & 10.2 (1.6) & 2.3 (1.1) \\
Stretch (e2) & 1.4 (0.3) & 11.0 (1.7) & 2.1 (0.5) \\
Stretch (e3) & 1.2 (0.4) & 10.8 (2.3) & 3.1 (1.9) \\
\midrule
DE (e1) & 1.1 (0.5) & 10.6 (1.8) & 5.3 (7.3) \\
DE (e2) & 1.4 (0.2) & 11.0 (1.5) & 2.0 (0.3) \\
DE (e3) & 1.3 (0.5) & 10.6 (1.2) & 4.9 (7.4) \\
\midrule
DEsnooker (e1) & 1.3 (0.2) & 10.9 (0.8) & 2.1 (0.4) \\
DEsnooker (e2) & 1.3 (0.1) & 11.3 (1.2) & 2.0 (0.3) \\
DEsnooker (e3) & 1.0 (0.4) & 12.1 (1.7) & 4.7 (6.3) \\
\midrule
\textbf{True value} & \textbf{1.0} & \textbf{0.01} &\textbf{10.0} \\
\bottomrule
\end{tabularx}
\end{table}


% Table 9
\begin{table}[ht]
\caption{Ensemble mean ($\mu$) and the standard deviation of the ensemble mean ($\sigma$), presented as mean (std). Results are from calibrating the parameters ($\theta$) of Model 3 with different samplers (Stretch, DE, DEsnooker), with each three ensembles (e). Five observations were used}
\label{tab_logbook8_obs5_model3}
\begin{tabularx}{\textwidth}{XXXX}
\toprule
 & $\theta_1$ & $\theta_2$ & $\theta_3$\\
\midrule
Stretch (e1) & 0.9 (0.1) & 0.1 (0.3) & 8.8 (1.5) \\
Stretch (e2) & 0.9 (0.2) & 0.2 (0.6) & 8.9 (1.5) \\
Stretch (e3) & 0.9 (0.1) & 0.4 (1.0) & 9.0 (1.6) \\
\midrule
DE (e1) & 0.7 (0.2) & 3.6 (8.3) & 6.5 (2.5) \\
DE (e2) & 0.9 (0.9) & 1.8 (2.8) & 5.7 (2.8) \\
DE (e3) & 0.6 (0.2) & 2.0 (3.5) & 6.9 (2.2) \\
\midrule
DEsnooker (e1) & 0.7 (0.2) & 1.1 (3.0) & 7.8 (2.0) \\
DEsnooker (e2) & 0.8 (0.2) & 0.7 (1.4) & 11.2 (8.8) \\
DEsnooker (e3) & 0.8 (0.2) & 0.5 (1.1) & 8.3 (2.0) \\
\midrule
\textbf{True value} & \textbf{1.0} & \textbf{0.01} &\textbf{10.0} \\
\bottomrule
\end{tabularx}
\end{table}



% Table 10
\begin{table}[ht]
\caption{Ensemble mean ($\mu$) and the standard deviation of the ensemble mean ($\sigma$), presented as mean (std). Results are from calibrating the parameters ($\theta$) of Model 4 with different samplers (Stretch, DE, DEsnooker), with each three ensembles (e). One observation was used}
\label{tab_logbook8_obs1_model4}
\begin{tabularx}{\textwidth}{lXXXXX}
\toprule
 & $\theta_1$ & $\theta_2$ & $\theta_3$ & $\theta_4$ & $\theta_5$ \\
\midrule
Stretch (e1) & 0.7 (1.1) & 3.9 (3.0) & 3.2 (3.5) & 6.0 (10.7) & 6.0 (6.0) \\
Stretch (e2) & 2.8 (2.5) & 3.5 (4.7) & 15.8 (11.7) & 10.2 (9.8) & 9.9 (7.9) \\
Stretch (e3) & 0.7 (0.9) & 9.7 (5.7) & 5.8 (8.6) & 7.3 (4.3) & 3.4 (3.4) \\
\midrule
DE (e1) & 0.8 (1.5) & 6.4 (6.4) & 6.0 (5.7) & 8.9 (4.3) & 4.7 (4.0) \\
DE (e2) & 0.8 (1.2) & 5.8 (3.5) & 5.2 (4.8) & 8.5 (5.6) & 5.3 (5.2) \\
DE (e3) & 2.3 (2.9) & 6.6 (6.5) & 8.2 (6.8) & 14.7 (8.6) & 8.0 (8.8) \\
\midrule
DEsnooker (e1) & 1.0 (1.7) & 4.1 (5.4) & 6.5 (6.3) & 8.8 (4.5) & 3.2 (3.2) \\
DEsnooker (e2) & 2.0 (3.1) & 5.9 (5.4) & 5.2 (3.6) & 8.4 (4.7) & 5.8 (8.6) \\
DEsnooker (e3) & 0.8 (1.3) & 4.8 (4.1) & 6.1 (6.5) & 8.5 (9.5) & 5.7 (4.7) \\
\midrule
\textbf{True value} & \textbf{1.0} & \textbf{0.1} &\textbf{4.0} & \textbf{0.01} &\textbf{3.0} \\
\bottomrule
\end{tabularx}
\end{table}

% Table 11
\begin{table}[ht]
\caption{Ensemble mean ($\mu$) and the standard deviation of the ensemble mean ($\sigma$), presented as mean (std). Results are from calibrating the parameters ($\theta$) of Model 4 with different samplers (Stretch, DE, DEsnooker), with each three ensembles (e). Three observations were used}
\label{tab_logbook8_obs3_model4}
\begin{tabularx}{\textwidth}{lXXXXX}
\toprule
 & $\theta_1$ & $\theta_2$ & $\theta_3$ & $\theta_4$ & $\theta_5$ \\
\midrule
Stretch (e1) & 0.3 (0.2) & 5.3 (2.2) & 6.9 (5.1) & 7.6 (11.2) & 9.2 (8.8) \\
Stretch (e2) & 2.9 (3.5) & 4.5 (4.2) & 11.9 (9.1) & 1.4 (2.9) & 5.2 (3.5) \\
Stretch (e3) & 0.4 (0.3) & 4.3 (5.9) & 3.3 (3.3) & 9.7 (7.6) & 7.1 (11.1) \\
\midrule
DE (e1) & 1.1 (1.2) & 6.5 (9.3) & 12.9 (10.7) & 15.0 (11.2) & 5.2 (4.1) \\
DE (e2) & 2.9 (5.0) & 6.2 (7.9) & 10.9 (9.6) & 8.6 (6.8) & 8.5 (7.8) \\
DE (e3) & 3.1 (5.0) & 6.1 (7.8) & 7.7 (8.8) & 12.2 (9.5) & 13.2 (13.1) \\
\midrule
DEsnooker (e1) & 0.8 (0.4) & 2.2 (2.4) & 5.3 (5.7) & 4.8 (4.0) & 2.7 (1.8) \\
DEsnooker (e2) & 1.0 (0.7) & 2.0 (1.9) & 5.0 (3.3) & 10.9 (7.9) & 3.6 (4.2) \\
DEsnooker (e3) & 1.4 (0.8) & 1.7 (1.9) & 2.2 (1.7) & 6.8 (6.9) & 7.2 (5.1) \\
\midrule
\textbf{True value} & \textbf{1.0} & \textbf{0.1} &\textbf{4.0} & \textbf{0.01} &\textbf{3.0} \\
\bottomrule
\end{tabularx}
\end{table}


% Table 12
\begin{table}[ht]
\caption{Ensemble mean ($\mu$) and the standard deviation of the ensemble mean ($\sigma$), presented as mean (std). Results are from calibrating the parameters ($\theta$) of Model 4 with different samplers (Stretch, DE, DEsnooker), with each three ensembles (e). Five observations were used}
\label{tab_logbook8_obs5_model4}
\begin{tabularx}{\textwidth}{lXXXXX}
\toprule
 & $\theta_1$ & $\theta_2$ & $\theta_3$ & $\theta_4$ & $\theta_5$ \\
\midrule
Stretch (e1) & 0.7 (0.6) & 8.7 (6.0) & 9.0 (5.2) & 10.9 (14.8) & 10.3 (9.1) \\
Stretch (e2) & 1.3 (0.4) & 0.5 (1.4) & 7.9 (12.2) & 0.1 (0.3) & 6.6 (5.4) \\
Stretch (e3) & 0.8 (0.2) & 2.9 (5.6) & 3.8 (2.2) & 0.7 (1.2) & 2.2 (1.1) \\
\midrule
DE (e1) & 0.9 (1.3) & 8.0 (11.1) & 14.3 (15.0) & 31.1 (24.3) & 15.4 (20.6) \\
DE (e2) & 2.9 (4.7) & 9.2 (10.9) & 15.4 (16.5) & 12.4 (11.2) & 7.7 (8.9) \\
DE (e3) & 2.3 (3.8) & 8.4 (8.9) & 12.0 (13.7) & 16.3 (11.4) & 7.3 (8.0) \\
\midrule
DEsnooker (e1) & 0.8 (0.9) & 8.1 (9.1) & 7.6 (10.1) & 15.8 (12.3) & 8.9 (11.9) \\
DEsnooker (e2) & 0.8 (0.8) & 1.4 (1.3) & 2.7 (2.8) & 16.5 (14.2) & 7.1 (5.4) \\
DEsnooker (e3) & 0.8 (0.7) & 7.3 (10.2) & 7.7 (11.0) & 15.3 (10.6) & 8.1 (9.9) \\
\midrule
\textbf{True value} & \textbf{1.0} & \textbf{0.1} &\textbf{4.0} & \textbf{0.01} &\textbf{3.0} \\
\bottomrule
\end{tabularx}
\end{table}

% Table 13
\begin{table}[ht]
\caption{Ensemble mean ($\mu$) and the standard deviation of the ensemble mean ($\sigma$), presented as $\mu (\sigma)$. Results are from calibrating the parameters ($\theta$) of Model 4 with one ensemble of 10 chains of DEsnooker. Fifty observations were used and chain length is 10000 steps including burn-in.}
\label{tab_logbook8_obs50_model4}
\begin{tabularx}{\textwidth}{lXXXXX}
\toprule
 & $\theta_1$ & $\theta_2$ & $\theta_3$ & $\theta_4$ & $\theta_5$ \\
\midrule
$\mu (\sigma)$ last 6000 steps & 0.9 (0.2) & 0.7 (1.1) & 3.0 (1.5) & 0.9 (1.4) & 2.9 (0.3) \\
$\mu (\sigma)$ last 3000 steps  & 0.9 (0.2) & 0.6 (1.0) & 3.6 (1.6) & 0.2 (0.4) & 2.5 (0.5) \\
\textbf{$\mu$ true value} & \textbf{1.0} & \textbf{0.1} &\textbf{4.0} & \textbf{0.01} &\textbf{3.0} \\
\midrule
$\hat{R}$ last 6000 steps & 1.58 & 1.60 & 1.55 & 1.58 & 1.27 \\
$\hat{R}$ last 3000 steps & 1.37 & 1.40 & 1.39 & 1.40 & 1.40 \\
\bottomrule
\end{tabularx}
\end{table}
